\documentclass[a4paper, fontsize=10pt]{scrartcl}
\usepackage[utf8]{inputenc}
\usepackage[ngerman]{babel}
\usepackage{stmaryrd}
\usepackage{amsfonts}
\usepackage{amsmath}
\usepackage{mathpazo} %schickere Schriftart für Text
\usepackage{MnSymbol} %schickere Symbole
\usepackage{graphicx}
\usepackage{amsthm}
\usepackage{shadethm}
\usepackage[all,2cell,ps]{xy}
\usepackage{setspace}
\usepackage{listings}
\usepackage{color}		%Farben ermöglichen
\usepackage{colortbl} %Tabellen mit Farbe ermöglichen
\usepackage{hyperref} %Hyperlinks einfügbarisieren
\usepackage{fancyvrb} %verbatim mit mehr Variationen
\usepackage{listings} %Quellcode mit listings einbinden
\usepackage{moreverb}
\usepackage{tocloft} %tableofcontent mit Optionen
\usepackage{ulem} %neue Befehle für Unterstriche
\usepackage{algpseudocode} 
\lstloadlanguages{Java} %Standardmäßig Java vorher laden
% Standard-Layout für die Code-Umgebung (alle Sprachen)
\lstset{%
      basicstyle=\footnotesize\ttfamily,
     showspaces=false,
     showtabs=false,
     columns=fixed,
     numbers=left,
     frame=none,
     numberstyle=\tiny,
     breaklines=true,
     showstringspaces=false,
     xleftmargin=1cm,
     tabsize=4
}%
\setlength\topmargin{-1cm}
\textheight 23cm
\textwidth 14cm
\setlength\oddsidemargin{1cm}
\setlength\evensidemargin{3.5cm}
\parindent=0pt
\setlength{\saveparindent}{\parindent}
%\pagestyle{empty}
\newcommand{\rem}[1]{} %rem-Kommentiermöglichkeit
\definecolor{dg}{rgb}{0.8,0.8,0.8}		%definiert dunkelgrau
\definecolor{hg}{rgb}{0.95,0.95,0.95}	%definiert hellgrau
%%%%%%%%%%%%%%%%%%%%%%%%%%%%%%%%%%%%%%%%%%%%%%%%%%%%%%%%%%%%%%%%%%%%%%%%%%%%%%%%%%%  
 
\begin{document} 

{\large Algorithmendesign \hfill Gruppe 2}\\  
{\large Lösungen zu Übungsblatt 2} \hfill Max Bannach\\
{\large WS 13/14}
\begin{flushright}Markus Richter (614027)\end{flushright}
\rule{\textwidth}{.3mm}

\section*{Aufgabe 2.1 Nichtoptimalität}
\subsection*{Teilaufgabe 1)}
Die Problemstellung besteht darin, dass einige Leute eine Wanderung entlang des "Appalachian Trail" machen wollen. Gegeben vieler im Vorfeld identifizierter Haltepunkte zum Campen wollen sie aus Angst nur bei Tageslicht wandern. Den Nachteilen dieser Einschränkung wollen sie mit folgendem Algorithmus entgegenwirken, der angeblich in einer minimalen Anzahl an Unterbrechungen resultiert:  \\
Jedes mal wenn sie an einer potentiellen Haltestelle zum Campen angelangt sind überprüfen sie, ob sie es vor Einbruch der Dunkelheit noch zum nächsten Haltepunkt schaffen. Wenn ja, so wandern sie weiter, andernfalls nicht.
\subsection*{Teilaufgabe 2)}
Ändert man die Problemstellung dahingehend, dass nach Einbruch der Dunkelheit gewandert werden darf, so berechnet der Algorithmus nicht mehr die optimale Lösung.\bigskip

Beispiel:\\
Angenommen innerhalb eines Intervalls mit der Distanz $d$ liegen 3 Haltepunkte $x_j, x_{j+1}, x_{j+2}$ sodass gilt: $d-x_j-x_{j+1}-x_{j+2}=0$. Da die Distanz zwischen den Haltestellen variieren kann, kann es passieren, dass kurz nach Erreichen von $x_{j+1}$, aber vor dem Erreichen von $x_{j+2}$ die Nacht hereinbricht. Nach dem Algorithmus wird die Gruppe gezwungen an der Stelle $x_{j+1}$ zu rasten. Ein Algorithmus ohne Beschränkung auf die Tageszeit würde aber erst eine Pause bei $x_{j+2}$ berechnen und somit zu einer geringeren Menge an Zwischenstopps führen. 
\newpage


\section*{Aufgabe 2.2 Konferenz-Organisation}

Der Lösungs-Algorithmus:

\begin{algorithmic}
\If {$i\geq maxval$}
    \State $i\gets 0$
\Else
    \If {$i+k\leq maxval$}
        \State $i\gets i+k$
    \EndIf
\EndIf
\end{algorithmic}

\begin{lstlisting}[mathescape]
Algorithmus Konferenz:
EINGABE: Menge $R$ von Jobs $r_j$ mit Laufzeiten $\rho(j)$, Deadlines $\delta(j)$ und schlechte Bewertungen $s(j)$
ordne die Jobs absteigend bzgl. ihrer schlechten Bewertung, d. h. $s(1)\geq ...s(n)$:
$t:=0$;
FOR $i=n$ TO $i=1$ DO
  $r:=0$;
  FOR $j=1$ TO $n$ DO
    IF($\delta(r_j)<=i$)
      THEN $r:=j$;
      Schleife beenden.
  IF(r != 0)
    THEN schedule $r_r$ fuer den Zeitraum $[t, t+\tau(r)[$; 
    $t:=t+\tau(r)$; 
    entferne $r_r$ aus $R$.
  OD.
WHILE $R \neq  \varnothing$ DO
  waehle $r_i \in R$;
  schedule $r_i$ fuer den Zeitraum $[t, t+\tau(i)[$;
  $t:=t+\tau(i)$;
  entferne $r_i$ aus $R$;
  OD;

  
\end{lstlisting}

Der Algorithmus basiert auf $n$ Zeitslots, die wiederum die Deadlines darstellen, wobei $n=$Anzahl der Jobs. Slot $n$ entspricht Deadline $n$. Es liegt eine absteigend nach schlechte Bewertung $s$ geordnete Liste vor. Der Algorithmus sucht inkremental in dieser Liste für jeden Slot einen Job, der mit seiner Deadline ausgeführt werden kann. Es kann also zunächst auch zu Lücken kommen, wenn Jobs mit großem $s$ Jobs mit kleinem $s$ verdrängen. Die verdrängen Jobs können ihre Deadline nicht einhalten und werden daher zum Schluss einfach nacheinander ausgeführt.\bigskip


\textbf{Korrektheit}\\
Ich zeige die Korrektheit des Algorithmus indem ich zeige, dass das Schedule $S$ keine Leerzeiten und keine Inversion aufweist.

\textbf{Leerzeiten}\\
Nach Konstruktion des Algorithmus sind alle Slots ab dem ersten nicht leeren Slot belegt und werden nacheinander ausgeführt. Es kommt also zu keinen Leerzeiten.

\textbf{Inversion}\\
Nach Konstruktion des Algorithmus wird immer der Jobs $r_i$ mit dem größten $s(i)$ ausgeführt. Inversion ist also nicht möglich

$\Longrightarrow$ Der Algorithmus ist korrekt.\bigskip

\textbf{Laufzeit}\\
Der Algorithmus sortiert anfangs eine Menge, was also zunächst zu einer Laufzeit von $\mathcal{O}(n\log n)$ führt. Weiterhin besteht der Algorithmus aus einer äußeren und einer inneren Schleife, wobei beide die Größe $n$ haben, dies führt zur Laufzeit $\mathcal{O}(n^2)$. Die while-Schleife wird linear durchlaufen, also $\mathcal{O}(n)$. Insgesamt beträgt die Laufzeit also $\mathcal{O}(n^2)$\bigskip


\textbf{Optimalität}\\
Es ist zu zeigen, dass die Anzahl der schlechten Bewertungen bei Anwendung dieses Algorithmus minimal ist.
\newpage

\section*{Aufgabe 2.3 Intervall-Abdeckung}

Der Lösungs-Algorithmus:
\begin{lstlisting}[mathescape]
Algorithmus Intervall-Abdeckung
EINGABE: Array A von Intervalen
Ordne das Array der Intervalle A aufsteigend nach dem linken Ende der Intervalle, insbesondere soll gelten: wenn left($a_i)=$left$(a_j$) und right$(b_j)>$right$(b_i)$, dann folgt $[$left$(a_i$),right$(b_i)]$ auf $[$left$(a_j$),right$(b_j)]$. 

$n:=$lenght$(A)$;
$B[1]:=S[1]$;
$j:=2$;
$k:=1$;
FOR $i=2$ TO $n$ DO
  IF right$(A[i])>$right$(A[k])$
    THEN B$[j]:=$A$[i]$;
    $j:=j+1$;
    $k:=i$;
RETURN B


\end{lstlisting}



\textbf{Korrektheit}\\
Der Algorithmus ist nach Konstruktion korrekt, da er zu jeder Eingabe eine korrekte Lösung liefert.\bigskip

\textbf{Laufzeit}\\
Zunächst erfolgt eine Sortierung mit einer Laufzeit von $\mathcal{O}(n\log n)$. Anschließend wird eine Schleife mit der Größe $n$ verarbeitet, d. h. mit linearer Laufzeit $\mathcal{O}(n)$. Daraus resultiert insgesamt eine Laufzeit von $\mathcal{O}(n\log n)$.


\end{document}

