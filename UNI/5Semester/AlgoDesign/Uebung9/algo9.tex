\documentclass[a4paper, fontsize=10pt]{scrartcl}
\usepackage[utf8]{inputenc}
\usepackage[ngerman]{babel}
\usepackage{stmaryrd}
\usepackage{amsfonts}
\usepackage{amsmath}
\usepackage{mathpazo} %schickere Schriftart für Text
\usepackage{MnSymbol} %schickere Symbole
\usepackage{graphicx}
\usepackage{amsthm}
\usepackage{shadethm}
\usepackage[all,2cell,ps]{xy}
\usepackage{setspace}

\usepackage{listings}
\usepackage{hyperref} %Hyperlinks einfügbarisieren
\usepackage{fancyvrb} %verbatim mit mehr Variationen
\usepackage{listings} %Quellcode mit listings einbinden
\usepackage{moreverb}
\usepackage{tocloft} %tableofcontent mit Optionen
\usepackage{tikz}
\usetikzlibrary{shapes}
\usepackage{algorithm}
\usepackage{algorithmic}
\usepackage{multicol}
\usepackage{soul}
\usepackage[makeroom]{cancel}


\definecolor{lightgray}{gray}{0.9}

\usetikzlibrary{arrows}
% \usepackage{ulem} %neue Befehle für Unterstriche
\usetikzlibrary{arrows,calc,shapes.multipart,chains}
\usetikzlibrary{calc,positioning}

\renewcommand{\algorithmicrequire}{\textbf{Input:}}
\renewcommand{\algorithmicensure}{\textbf{Output:}}

\lstloadlanguages{Java} %Standardmäßig Java vorher laden
% Standard-Layout für die Code-Umgebung (alle Sprachen)
\lstset{%
      basicstyle=\small\ttfamily,
     showspaces=false,
     showtabs=false,
     columns=fixed,
     numbers=left,
     frame=none,
     numberstyle=\tiny,
     breaklines=true,
     showstringspaces=false,
     xleftmargin=1cm,
     tabsize=4
}%
\setlength\topmargin{-1cm}
\textheight 23cm
\textwidth 14cm
\setlength\oddsidemargin{1cm}
\setlength\evensidemargin{3.5cm}
\parindent=0pt
\setlength{\saveparindent}{\parindent}
%\pagestyle{empty}
\newcommand{\rem}[1]{} %rem-Kommentiermöglichkeit
\definecolor{dg}{rgb}{0.8,0.8,0.8}		%definiert dunkelgrau
\definecolor{hg}{rgb}{0.95,0.95,0.95}	%definiert hellgrau
%%%%%%%%%%%%%%%%%%%%%%%%%%%%%%%%%%%%%%%%%%%%%%%%%%%%%%%%%%%%%%%%%%%%%%%%%%%%%%%%%%%  
 
\begin{document} 

{\huge{Algorithmendesign} \hfill \large{ Gruppe 2}}\\  
{\large Lösungen zu Übungsblatt 9} \hfill Max Bannach\\
{\large WS 13/14}
\begin{flushright}Markus Richter (614027)\end{flushright}
\rule{\textwidth}{.3mm}

\section*{Aufgabe 9.3 Kuckucks-Hashing}
\subsection*{Teilaufgabe 1)}


\begin{center}

    \begin{multicols}{2}
    $h(x)=((11x+14)\mod 17)\mod 3$\smallskip
   \vfill
   \columnbreak
   
    $h'(x)=((x+10)\mod 17)\mod 3$\smallskip
    
  \end{multicols}
  
  \begin{multicols}{2}
    \begin{tabular}{r|rrrrr}
      &3 & 5 & 2 & 7 & 10\\
      \hline\\
      0 \\
      1 &3\\
      2 \\
    \end{tabular}
  \bigskip

    \begin{tabular}{r|rrrrr}
      &3 & 5 & 2 & 7 & 10\\
      \hline\\
      0 \\
      1 \\
      2 \\
    \end{tabular}
  \end{multicols}
  
    \begin{multicols}{2}
    \begin{tabular}{r|rrrrr}
      &3 & 5 & 2 & 7 & 10\\
      \hline\\
      0 \\
      1 &3&3\\
      2 \\
    \end{tabular}
  \bigskip

    \begin{tabular}{r|rrrrr}
      &3 & 5 & 2 & 7 & 10\\
      \hline\\
      0 & & 5\\
      1 \\
      2 \\
    \end{tabular}
  \end{multicols}
  
     \begin{multicols}{2}
    \begin{tabular}{r|rrrrr}
      &3 & 5 & 2 & 7 & 10\\
      \hline\\
      0 \\
      1 &3&3&3\\
      2 &&&2
    \end{tabular}
  \bigskip

    \begin{tabular}{r|rrrrr}
      &3 & 5 & 2 & 7 & 10\\
      \hline\\
      0 & & 5&5\\
      1 \\
      2 \\
    \end{tabular}
  \end{multicols}
  
      \begin{multicols}{2}
    \begin{tabular}{r|rrrrr}
      &3 & 5 & 2 & 7 & 10\\
      \hline\\
      0&  & & &   7\\
      1&  3 & 3 & 3  &  3\\
      2&  & & 2 & 2
    \end{tabular}
  \bigskip

    \begin{tabular}{r|rrrrr}
      &3 & 5 & 2 & 7 & 10\\
      \hline\\
      0 & & 5&5&5\\
      1 \\
      2 \\
    \end{tabular}
  \end{multicols}
  
  Kollision!
  
     \begin{multicols}{2}
    \begin{tabular}{r|rrrrr}
      &3 & 5 & 2 & 7 & 10\\
      \hline\\
      0&  & & &   7 & 7\\
      1&  3 & 3 & 3  &  3 & 3\\
      2&  & & 2 & 2 & \bcancel{2} 10
    \end{tabular}
  \bigskip

    \begin{tabular}{r|rrrrr}
      &3 & 5 & 2 & 7 & 10\\
      \hline\\
      0 & & 5&5&5&5\\
      1 \\
      2 \\
    \end{tabular}
  \end{multicols}
  
       \begin{multicols}{2}
    \begin{tabular}{r|rrrrr}
      &3 & 5 & 2 & 7 & 10\\
      \hline\\
      0&  & & &   7&7\\
      1&  3 & 3 & 3 &3 &  3\\
      2&  & & 2 & 2 & 10
    \end{tabular}
  \bigskip

    \begin{tabular}{r|rrrrr}
      &3 & 5 & 2 & 7 & 10\\
      \hline\\
      0 & & 5&5&5&\bcancel{5} 2\\
      1 \\
      2 \\
    \end{tabular}
  \end{multicols}
  
      \begin{multicols}{2}
    \begin{tabular}{r|rrrrr}
      &3 & 5 & 2 & 7 & 10\\
      \hline\\
      0&  & & &   7 & 7\\
      1&  3 & 3 & 3  &  3 & \bcancel{3} 5\\
      2&  & & 2 & 2 & 10
    \end{tabular}
  \bigskip

    \begin{tabular}{r|rrrrr}
      &3 & 5 & 2 & 7 & 10\\
      \hline\\
      0 & & 5&5&5&2\\
      1 &&&&&\\
      2 \\
    \end{tabular}
  \end{multicols}
  
        \begin{multicols}{2}
    \begin{tabular}{r|rrrrr}
      &3 & 5 & 2 & 7 & 10\\
      \hline\\
      0&  & & &   7 & 7\\
      1&  3 & 3 & 3  &  3 & 5\\
      2&  & & 2 & 2 & 10
    \end{tabular}
  \bigskip

    \begin{tabular}{r|rrrrr}
      &3 & 5 & 2 & 7 & 10\\
      \hline\\
      0 & & 5&5&5&2\\
      1 &&&&&3\\
      2 \\
    \end{tabular}
  \end{multicols}
  
  \end{center}
  
\subsection*{Teilaufgabe 2}

Sei zufällig folgendes gegeben:\bigskip

\begin{center}
    $h(x)=x\mod 11 $\smallskip

    \begin{tabular}{r|rrrrrrrrrr}
      &20 & 50 & 53 & 75 & 100 & 67 & 105 & 3 & 36 & 39\\
      \hline\\
      0 \\
      1 &&&&&100&67&67&67&67&100\\
      2 \\
      3 &&&&&&&&3&3&36\\
      4 \\
      5 \\
      6 &&50&50&50&50&50&50&50&50&50\\
      7 \\
      8 \\
      9 &20&20&20&20&20&20&53&53&53&75\\
      10 \\
    \end{tabular}
  \bigskip

 $h'(x)=\left\lfloor \dfrac{k}{11}\right\rfloor \mod 11$\smallskip
 
      \begin{tabular}{r|rrrrrrrrrr}
      &20 & 50 & 53 & 75 & 100 & 67 & 105 & 3 & 36 & 39\\
      \hline\\
      0 &&&&&&&&&&3\\
      1 &&&&&&&20&20&20&20\\
      2 \\
      3 &&&&&&&&&36&39\\
      4 &&&53&53&53&53&50&50&50&53\\
      5 \\
      6 &&&&75&75&75&75&75&75&67\\
      7 \\
      8 \\
      9 &&&&&&100&100&100&100&105\\
      10 \\
    \end{tabular}
    \newpage
    
    
  Fügt man nun eine 6 ein gerät man in einen Zyklus:\bigskip
  
    \begin{tabular}{r|rr|rr}
    Schlüssel & h(x) && h'(x)\\
     & alter Wert & neuer Wert & alter Wert & neuer Wert\\
     \hline\\
     6&50&6&53&50\\
     53&75&53&67&75\\
     67&100&67&105&100\\
     105&6&105&3&6\\
     3&36&3&39&36\\
     39&105&39&100&105\\
     100&67&100&75&67\\
     75&53&75&50&53\\
     50&39&50&36&39\\
     36&3&36&6&3\\
     6&50&6&53&50
     
    \end{tabular}
\end{center}
    

\section*{Aufgabe 9.4 Wir bauen einen Flughafen!}

\subsection*{Teilaufgabe 1}
Das Optimierungsproblem sei $\Pi$ genannt, wobei für eine Eingabe $X=x_1,\dots, x_n$ eine Folge von Aktionen $Y=y_1,\dots,y_n$ zu bestimmten ist. Sei $X$ in diesem Beispiel eine binäre Folge, die mit einem Block von Einsen beginnt, welche die Jahre repräsentieren, in denen noch gebaut wird. Wenn der Flughafen zum Zeitpunkt $x_i=1$ fertiggestellt wird wechselt die Folge ab $x_{i+1}$ auf $0$. Die Ausgabe $Y$ gibt an, ob bei einem $x_i=1$ die Aktion \textsc{Kauf} durchgeführt wird, falls nicht bereits in der Vergangenheit geschehen, oder stattdessen die Aktion $y_i=\textrm{\textsc{MIET}}$. \smallskip

Um das Moor trockenzulegen, kann man die Pumpe
  \begin{enumerate}
  \item kaufen, wofür einmalig $K= 980.000$ Euro anfallen, wovon aber sogleich $0,5\cdot K$ also $490.000$ Euro, wieder abgezogen werden, da das Gerät veräußert wird. Effektiv beträgt der Preis demnach $490.000$ Euro.
  \item mieten, wofür pro Jahr $L=70.000$ Euro anfallen.
  \end{enumerate} \smallskip
  
$S_r$ beschreibt die möglichen Strategien, wobei $r=0,1,\dots$ die Jahre $r$ mieten bedeutet und, falls im Jahr $r+1$ noch gebaut wird, danach einmalig kaufen.\smallskip

Bezeichnet $t(X)$ die Anzahl der Einsen in $X$, so berechnen sich die Kosten von $S_r$ analog zum Beispiel aus dem Skript als:\bigskip

$cost(S_r) = \begin{cases}

t(X)\cdot L&\textrm{falls $t(X)\leq r$},\\
r\cdot L+K-0,5\cdot K & \textrm{sonst.}
        
  \end{cases}      $

\subsection*{Teilaufgabe 2}
Die optimale Strategie ist ebenfalls analog zum Skript -- nur dass der Kaufpreis effektiv $0,5\cdot K$ also $490.000$ Euro beträgt -- wie folgt\bigskip

$cost(\textsc{OPT}) = \begin{cases}

t(X)\cdot L&\textrm{falls $t(X)\leq \frac{0,5\cdot K}{L}$},\\
r\cdot L+K-0,5\cdot K & \textrm{sonst.}
        
  \end{cases}$\smallskip
  
\newpage
Das Verhältnis der Kosten von $S_r$ zu den minimalen Kosten beträgt\bigskip


$\dfrac{cost(S_r)}{cost(OPT)} = \begin{cases}

    \dfrac{t(X)\cdot L}{t(X)\cdot L}=1                  &\textrm{falls $t(X)\leq \min \left \{\frac{0,5\cdot K}{L} \right \}$},\\[5mm]

    \dfrac{t(X)\cdot L}{0,5\cdot K}                    & \textrm{falls $\frac{0,5\cdot K}{L}< t(X)\leq r$,} \\[5mm]

    \dfrac{r\cdot L + 0,5\cdot K}{t(X)\cdot L}         & \textrm{falls $r<t(X)\leq \frac{0,5\cdot K}{L}$,} \\[5mm]

    \dfrac{r\cdot L+0,5\cdot K}{0,5\cdot K}            & \textrm{falls $t(X) > \min \left \{ \frac{0,5\cdot K}{L},r \right \} $.}
            
    \end{cases}$
    
Wir wollen nun den Parameter $r$ finden -- denn nur darauf haben wir Einfluss -- für den sich die Strategie $S_r$ ergibt, die den minimal schlimmen Worst-Case im Verhältnis zum optimalen Fall liefert, d. h. der Ausdruck \smallskip

$\max\left \{ \dfrac{r\cdot L}{K\cdot 0,5}, \dfrac{r\cdot L + K\cdot 0,5}{(r+1)\cdot L}, \dfrac{r\cdot L+ K \cdot 0,5}{K\cdot 0,5} \right \}$ soll minimiert werden. Der dritte Term dominiert den ersten, d. h. es genügt den zweiten und dritten Term zu betrachten. Setzt man die beiden Terme gleich und rechnet nach $r$ um, so ergibt sich:
$r=\dfrac{0,5\cdot K}{L}-1$

Setzt man diesen Wert in die oben genannten Terme ein ergibt sich analog zum Ski-Problem der Quotient $2-\dfrac{L}{0,5\cdot K}$. Der zweite Fall ist nicht relevant, weil hierfür $2$ raus kommt, die immer größer ist als der oben genannte Quotient.\\

Daraus folgt, dass konkret in diesem Beispiel, $r=\dfrac{490.000}{70.000}-1=6$ Jahre die bestmögliche Strategie ergibt, sofern man kein vorheriges Wissen bzgl. der Bauzeit hat. Diese Lösung ist maximal $2-\dfrac{70.000}{490.000}=1,8571$, also $0,8571$ mal schlechter als die optimale Strategie.
 
\end{document}
