% This is an example to show the beamer template NewLuebeck.
\documentclass[9pt]{beamer}

% Alternatively you can use the article mode to print a handout
%\documentclass{article}
%\usepackage{beamerarticle}

\usepackage[T1]{fontenc}                    % Ausgabekodierung
\usepackage[utf8]{inputenc}                 % Eingabekodierung
\usepackage[english]{babel}               % English 

\mode<presentation>{
\usetheme{NewLuebeck}
\title{Web Servers}
\author[Richter]{Markus Richter}
\institute{Institut für Softwaretechnik und Programmiersprachen}
\subject{Beispiel}
\keywords{Web servers}
}

\mode<article>{
\usepackage{document}
\documentAuthors{Markus Richter}
\documentTerm{WS 2013}
\documentDate{\today}
\documentTitle{Web Servers}
}

\begin{document}

\frame{\titlepage}
\mode<article>{\printDocumentHeader}

\begin{frame}
\frametitle<presentation>{Table of Contents}
\tableofcontents
\end{frame}

\section{Introduction}

\begin{frame}
\frametitle<presentation>{Introduction}
  \begin{itemize}
    \item The Internet is growing
    \item Not only for information gain
    \item Business, communication, entertainment
    \item Key to the success of the internet
    \item Made possible by Web servers
  \end{itemize}
\end{frame}

\section{Beispiele}

\begin{frame}
\frametitle<presentation>{Itemize/Enumerate}
\begin{itemize}
  \item Item1
  \item Item2
  \item Item3
\end{itemize}
\begin{enumerate}
  \item Enum1
  \item Enum2
  \item Enum3
\end{enumerate}
\end{frame}

\begin{frame}
\frametitle<presentation>{Blocks}
\begin{block}{Definition}
Dies ist eine Definition!
\end{block}
\begin{exampleblock}{Beispiel}
Dies ist ein Beispiel!
\end{exampleblock}
\begin{alertblock}{Warnung}
Dies ist eine Warnung!
\end{alertblock}
\end{frame}

\begin{frame}<presentation>
\frametitle{Only visible in presentation mode}
This slide is only visible in presentation mode
\end{frame}

\mode<article>{
This text is only visible in article mode.
}

\begin{frame}
\frametitle<presentation>{TikZ}
\begin{center}
\begin{tikzpicture}[scale=1,auto,swap]
  \clip (0,0) rectangle (8, 5);
  \draw[color=oceangreen!20!white] (0,0) rectangle (8, 5);
  \node[box] (a) at (2, 4) {%
    \parbox{3cm}{Beispiel\newline$1+1=2$}};
  \node[node] (b) at (6, 4) {$x$};
  \node[box] (c) at (6, 2.5) {%
    \parbox{3cm}{$x+x=2x$}};
  \node[node] (d) at (6, 1) {$y$};
  \node[box] (e) at (2, 1) {%
    \parbox{3cm}{Ergebnis\newline$xy+xy=2xy$}};
  \path[draw,->] (a) -- (b);
  \path[draw,->] (b) -- (c);
  \path[draw,->] (c) -- (d);
  \path[draw,->] (d) -- (e);
\end{tikzpicture}
\end{center}
\end{frame}

\section{Ende}

\begin{frame}
\frametitle<presentation>{Ende}
Ich hoffe, diese Vorlage gefällt Ihnen!
\end{frame}

\end{document}
