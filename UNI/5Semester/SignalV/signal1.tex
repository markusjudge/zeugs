\documentclass[a4paper, fontsize=10pt]{scrartcl}
\usepackage[utf8]{inputenc}
\usepackage[ngerman]{babel}
\usepackage{stmaryrd}
\usepackage{multicol}
\usepackage{amsfonts}
\usepackage{amsmath}
\usepackage{mathpazo} %schickere Schriftart für Text
\usepackage{MnSymbol} %schickere Symbole
\usepackage{graphicx}
\usepackage{amsthm}
\usepackage{shadethm}
\usepackage[all,2cell,ps]{xy}
\usepackage{setspace}
\usepackage{listings}
\usepackage{color}		%Farben ermöglichen
\usepackage{colortbl} %Tabellen mit Farbe ermöglichen
\usepackage{hyperref} %Hyperlinks einfügbarisieren
\usepackage{fancyvrb} %verbatim mit mehr Variationen
\usepackage{listings} %Quellcode mit listings einbinden
\usepackage{moreverb}
\usepackage{tocloft} %tableofcontent mit Optionen
\usepackage{ulem} %neue Befehle für Unterstriche
\usepackage[style=1]{mdframed} 

\lstloadlanguages{Java} %Standardmäßig Java vorher laden
% Standard-Layout für die Code-Umgebung (alle Sprachen)
\lstset{%
      basicstyle=\footnotesize\ttfamily,
     showspaces=false,
     showtabs=false,
     columns=fixed,
     numbers=left,
     frame=none,
     numberstyle=\tiny,
     breaklines=true,
     showstringspaces=false,
     xleftmargin=1cm,
     tabsize=4
}%
\setlength\topmargin{-1cm}
\textheight 23cm
\textwidth 14cm
\setlength\oddsidemargin{1cm}
\setlength\evensidemargin{3.5cm}
\parindent=0pt
\setlength{\saveparindent}{\parindent}
%\pagestyle{empty}
\newcommand{\rem}[1]{} %rem-Kommentiermöglichkeit
\definecolor{dg}{rgb}{0.8,0.8,0.8}		%definiert dunkelgrau
\definecolor{hg}{rgb}{0.95,0.95,0.95}	%definiert hellgrau
%%%%%%%%%%%%%%%%%%%%%%%%%%%%%%%%%%%%%%%%%%%%%%%%%%%%%%%%%%%%%%%%%%%%%%%%%%%%%%%%%%%  
 
\begin{document} 
\mdfsetup{outerlinewidth=3pt,innerlinewidth=0pt,outerlinecolor=red,topline=false,rightline=false,bottomline=false} 

{\large Signalverarbeitung \hfill Gruppe NUMMER}\\  
{\large Lösungen zu Übungsblatt 1} \hfill ÜBUNGSLEITERNAME\\
{\large WS 13/14}
\begin{flushright}Markus Richter, Mats Pfeiffer\end{flushright}
\rule{\textwidth}{.3mm}

\section*{Aufgabe 2}
\subsection*{Teilaufgabe a)}
\begin{align*}
\cos+j\sin=irgendas
\\\Leftrightarrow j\sin+\cos=auch irgendwas
\end{align*}
\section*{Aufgabe 3}
$ x_1(n)=(-1)^n $\\
Divergent gegen $\pm1$, daher periodisch. Die kleinste Periode $T=2$, da $x_1(n)=x_1(n+2)$\bigskip

$ x_2(n)=j^n$\\
Imaginäre Zahl $j, j^2=-1$, daher kleinste Periode $T=2$\bigskip

$x_4(t)=2\sin(\pi t)+4\sin\left( \dfrac{2\pi t}{3}\right)$\\
Funktion besteht aus Summe von $\sin$-Funktionen, die $2\pi $-periodisch sind. Daher ist die Summe ebenfalls periodisch.\\
$\Rightarrow 2\sin (\pi t)$: $T_a=2$.\\$\Rightarrow 4\sin\left( \dfrac{2\pi t}{3}\right)$: $T_b=\dfrac{2\pi}{\dfrac {2\pi}{3}}=3$. Der größte gemeinsame Teiler ist $6$, daher $T=6$\bigskip

$x_4(t)=2\sin(\pi n)+4\sin\left( \dfrac{2\pi n}{3}\right)$. Analog zu $x_3$, der eingeschränkte Definitionsbereich ändert hier nichts an der Periodizität, da $\pi$  vorgegeben ist.\bigskip

$x_5(t)=4\sin (t)+5\sin (3t)$\\
Wieder Summe von zwei $2\pi$-periodischen $\sin$-Funktionen. Daher ebenfalls periodisch.\\
$4\sin (t):$ gilt offensichtlich $T_a=2\pi$.\\$5\sin (3t): T_b=\dfrac{2\pi}{3} \Rightarrow T=2\pi$\bigskip

$x_6(n)=5\sin (n)+6\sin (3n)$.\\
Die kleinste Periode $T$ wäre hier auch $2\pi $ allerdings gilt $\pi \notin \mathbb{Z}\Rightarrow$ Keine Periodizität. 


\section*{Aufgabe 4}
\subsection*{a)}
$(1+j)^2=1^2+2j+j^2=2j$\\
$r=\sqrt{2^2}=2$, $\varphi_c=\dfrac{\pi}{2}$ $\Rightarrow$ Polare Form: $2e^{j\pi/3}$\bigskip

\subsection*{b)}
$2e^{j\pi/3}\cdot e^{j\pi /6}=2e^{j\pi /3 + j\pi/6}=2e^{j\pi/2} $\\
Kartesische Form: $2\cos\left(\dfrac{\pi}{2}\right)+2j\sin\left(\dfrac{\pi}{2}\right)=2j$\bigskip 

\subsection*{c)}
$(-1+j\sqrt{3})^{13}$\\
$r=\sqrt{(-1)^2+(\sqrt{3})^2}=2$, $\varphi_c=\arctan\left(\dfrac{\sqrt{3}}{-1}\right)+\pi=\dfrac{2}{3}\pi$\\
$\Rightarrow$ Polare Form: $\left(2e^{j2\pi/3}\right)^{13}=2^{13}\cdot e^{j16\pi/3}$\bigskip 

\subsection*{d)}
$\left(-1+j\sqrt{3}\right)^{-1}=\dfrac{1}{-1+j\sqrt{3}}=\dfrac{-1-j\sqrt{3}}{\left(-1+j\sqrt{3}\right)\left(-1-j\sqrt{3}\right)}=\dfrac{-1-\sqrt{3}}{4}=-\dfrac{1}{4}\left(1+j\sqrt{3}\right)$\\
Polare Form analog zu c):
$\left(2e^{j2\pi/3}\right)^{-1}=\dfrac{1}{2}\cdot e^{-j2\pi/3}$

\section*{Aufgabe 5}

$x(t)=2+t+t2-2t^3$\\
$x_g(t)=\dfrac{\left(2+t+t^2-2t^3\right)+\left(2-t+(-t)^2+2t^3\right)}{2}=\dfrac{4+2t^2}{2}=2+t^2$\\
$x_u(t)=\dfrac{\left(2+t+t^2-2t^3\right)-\left(2-t+t^2+2t^3\right)}{2}=t-2t^3$\bigskip


$x(t)=1+t^3\cos (t)+t\sin (2t)$\\
\begin{mdframed}
Anmerkung:\\
$\cos(t)=cos(-t)$\\
$\sin(t)=-\sin(-t)$\\
\end{mdframed} 

$x_g(t)=\dfrac{\left(1+t^3\cos (t)+t\sin (2t)\right)+\left(1-t^3\cos(-t)-t\sin (-2t)\right)}{2}$\\
$=\dfrac{2+2t\sin(2t)}{2}=1+t\sin(2t)$\\
$x_u(t)=\dfrac{\left(1+t^3\cos (t)+t\sin(2t)\right)-\left(1+t^3\cos(-t)+t\sin (-2t)\right)}{2}=\dfrac{2t^3\cos(t)}{2}=t^3\cos(t)$\bigskip

$x(t)=j+jt+\left(jt)\right)^2$\\
$x_g(t)=\dfrac{j+jt-t^2+\left(-j+jt-t^2\right)}{2}=\dfrac{-2t^2+2t}{2}=-t^2+jt$\\
$x_u(t)=\dfrac{j+jt-t^2-\left(-j+jt-t^2\right)}{2}=\dfrac{2j}{2}=j$\bigskip

$x(t)=1-e^{jt}+e^{jt^2}$\\
$x_g(t)=\dfrac{1-e^{jt}+e^{jt^2}+\left(1-e^{-j-t}+e^{-jt^2}\right)}{2}=$\\
\begin{mdframed}
NR: $e^{jt^2}+e^{-jt^2}\Leftrightarrow e^{jt^2}-e^{j-t^2}\Leftrightarrow \left(\cos(t^2)+j\sin(t^2)\right)+\left(\cos(-t^2)+j\sin(-t^2)\right)=$\\
$=2\cos(t^2)+0$
\end{mdframed} 
$=\dfrac{2-2e^{jt}+2\cos(t^2)}{2}=1-e^{jt}+\cos(t^2)$
$x_u(t)=\dfrac{1-e^{jt}+e^{jt^2}-\left(1-e^{-j-t}+e^{-jt^2}\right)}{2}=$\\
$=\dfrac{2j\sin(t)+2j\sin(t^2)}{2}=j\sin(t^2)+j\sin(t)$

\end{document}
