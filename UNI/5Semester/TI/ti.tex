\documentclass[a4paper, fontsize=10pt]{scrartcl}
\usepackage[utf8]{inputenc}
\usepackage[ngerman]{babel}
\usepackage{stmaryrd}
\usepackage{amsfonts}
\usepackage{amsmath}
\usepackage{mathpazo} %schickere Schriftart für Text
\usepackage{MnSymbol} %schickere Symbole
\usepackage{graphicx}
\usepackage{amsthm}
\usepackage{shadethm}
\usepackage[all,2cell,ps]{xy}
\usepackage{setspace}
\usepackage{listings}
\usepackage{color}		%Farben ermöglichen
\usepackage{colortbl} %Tabellen mit Farbe ermöglichen
\usepackage{hyperref} %Hyperlinks einfügbarisieren
\usepackage{fancyvrb} %verbatim mit mehr Variationen
\usepackage{listings} %Quellcode mit listings einbinden
\usepackage{moreverb}
\usepackage{tocloft} %tableofcontent mit Optionen
\usepackage{ulem} %neue Befehle für Unterstriche
\usepackage{multicol}
\usepackage{wasysym} %Für \checked (Haken)
\usepackage{float}%Um Positionierung von Abbildungen zu erzwingen.
\lstloadlanguages{Java} %Standardmäßig Java vorher laden
% Standard-Layout für die Code-Umgebung (alle Sprachen)
\lstset{%
      basicstyle=\footnotesize\ttfamily,
     showspaces=false,
     showtabs=false,
     columns=fixed,
     numbers=left,
     frame=none,
     numberstyle=\tiny,
     breaklines=true,
     showstringspaces=false,
     xleftmargin=1cm,
     tabsize=4
}%
\setlength\topmargin{-1cm}
\textheight 23cm
\textwidth 14cm
\setlength\oddsidemargin{1cm}
\setlength\evensidemargin{3.5cm}
\parindent=0pt
\setlength{\saveparindent}{\parindent}
%\pagestyle{empty}
\newcommand{\rem}[1]{} %rem-Kommentiermöglichkeit
\definecolor{dg}{rgb}{0.8,0.8,0.8}		%definiert dunkelgrau
\definecolor{hg}{rgb}{0.95,0.95,0.95}	%definiert hellgrau
%%%%%%%%%%%%%%%%%%%%%%%%%%%%%%%%%%%%%%%%%%%%%%%%%%%%%%%%%%%%%%%%%%%%%%%%%%%%%%%%%%%  
 
\begin{document} 

{\large Theoretische Informatik \hfill Gruppe 3}\\  
{\large Lösungen zu Übungsblatt 2} \hfill Florian Thaeter\\
{\large WS 13/14}
\begin{flushright}Felix Bayer (607241), Nico Kohlmorgen (607490),\\ Justin Neumann (617626), Markus Richter (614027)\end{flushright}
\rule{\textwidth}{.3mm}

\section*{Aufgabe 2.1 Ableitungen angeben}
S$\Rightarrow _G$aS$\Rightarrow _G$aaS$\Rightarrow _G$aaaA$\Rightarrow _G$aaabB$\Rightarrow _G$aaabaB$\Rightarrow _G$aaabaa

\section*{Aufgabe 2.2 Korrektheitsbeweis einer Grammatik durchführen}
\subsection*{1}
S$\Rightarrow _G$A00A$\Rightarrow _G$A000A$\Rightarrow _G$A0001A$\Rightarrow _G$A0001$\lambda\Rightarrow _G$1A0001$\lambda\Rightarrow _G$1$\lambda$0001$\lambda\Rightarrow _G$10001

\newpage

\subsection*{2}
Die Grammatik erzeugt die Sprache:\\
$L=\left\{w\in(0,1)^*\rvert w=u00v \textrm{ für }u\in (0,1)^*, v\in (0,1)^*\right\}$\bigskip

Sei $L=\left\{w\in(0,1)^*\rvert w=u00v \textrm{ für }u\in (0,1)^*, v\in (0,1)^*\right\}$. Wir müssen zeigen, dass $L(G)=L$ gilt. Dazu zeigen wir zwei Richtungen: $L\subseteq L(G)$ und $L(G)\subseteq  L$.\bigskip

Wir beginnen mit $L\subseteq L(G)$. Sei $w\in L$ beliebig. Dann gilt\\$w=u00v \textrm{ für ein }u\in (0,1)^*, \textrm{ und ein }v\in(0,1)^*$. Wenn man nun auf das Startsymbol genau einmal die Regel $S\rightarrow A00A$ anwendet und dann $n$-mal die Regel $A\rightarrow 1A$ und/oder $n$-mal die Regel $A\rightarrow0A$ und dann zweimal die Regel $A\rightarrow\lambda$, so ergeben sich folgende Ableitungsketten:\bigskip

\textbf{Fall 1:}\\
$S\Rightarrow A00A\Rightarrow ...\Rightarrow uA00vA\Rightarrow uA00v\Rightarrow...\Rightarrow u00v$\bigskip

\textbf{Fall 2:}\\
$S\Rightarrow A00A\Rightarrow ...\Rightarrow uA00vA\Rightarrow u00vA\Rightarrow ...\Rightarrow u00v$\bigskip

Also gilt $S\Rightarrow^* u00v=w$. Damit gilt auch $w\in L(G)$ und somit $L\subseteq L(G)$.\bigskip

Nun bleibt $L(G)\subseteq L$ zu zeigen. Sei dazu $w\in L(G)$ beliebig. Dann gilt $S\Rightarrow^* w$ über eine Ableitungskette\\
$s=w_1\Rightarrow...\Rightarrow w_n = w$.\\
Das Wort $w_1$ enthält ein Nonterminal, nämlich S. Bei jeder Regelanwendung kann die Anzahl der Nonterminale nicht wachsen --- die rechten Regelseiten enthalten immer nur zwei Nonterminale. Demnach enthalten alle $w_i$ höchstens zwei Nonterminale. Es gilt nach Definition der erzeugten Sprache $w\in (0,1)^*$, d. h. $w_n$ enthält keine Nonterminale. Daraus folgt, dass genau im letzten Ableitungsschritt die Regel $A\rightarrow \lambda$ zum zweiten Mal angewandt worden sein muss. Folglich lautete die Ableitung:\bigskip

\textbf{Fall 1:}\\
$S\Rightarrow A00A\Rightarrow ...\Rightarrow uA00vA \Rightarrow u00vA \Rightarrow ... \Rightarrow u00v$\bigskip

\textbf{Fall 2:}\\
$S\Rightarrow A00A \Rightarrow ... \Rightarrow uA00vA \Rightarrow uA00v \Rightarrow ... \Rightarrow u00v$\bigskip

Also gilt $w=u00v$ und somit $w\in L$ und somit $L(G)\subseteq L$.
\begin{flushright}$\blacksquare$\end{flushright}
\newpage


\section*{Aufgabe 2.3 Grammatiken angeben}
\subsection*{1}
\begin{multicols}{2}
$S\Rightarrow 0A\lvert 1A$\\
$A\Rightarrow 00A\lvert 10A\lvert 01A \lvert 11A\lvert \lambda$\columnbreak

Dadurch, dass ein Wort anfänglich immer die Länge 1 hat danach nur um zwei Terminale erweitert werden kann, ist die Länge immer ungerade.
\end{multicols}

\subsection*{2}
\begin{multicols}{2}
$S\Rightarrow 1S\lvert 00A\lvert 0S$\\
$A\Rightarrow 0A\lvert 1A\lvert \lambda$\columnbreak

Das Beenden der Ableitung ist nur möglich über den Schritt $S\Rightarrow00A$, damit wird garantiert, dass $00$ als Teilwort vorkommt.
\end{multicols}

\subsection*{3}
\begin{multicols}{2}
$S\Rightarrow 0S\lvert A$\\
$A\Rightarrow 1A\lvert \lambda$\columnbreak

Nach einer 1 kann nach Definition keine 0 folgen. 
\end{multicols}

\subsection*{4}
\begin{multicols}{2}
$S\Rightarrow 0S\lvert A$\\
$A\Rightarrow 1A\lvert \lambda$\columnbreak

Im Grunde dasselbe wie oben.
\end{multicols}

\section*{Aufgabe 2.4 Regelart angeben}

\begin{table}[H]
  \label{tab:}

  \begin{center}
    \begin{tabular}{rccc}
        Nr. & regulär & kontextfrei & kontextsensitiv \\
        1 & \checked & \checked & X\\
        2 & \checked & \checked & X\\
        3 & X & X & X\\
        4 & \checked & \checked & \checked\\
        5 & X & \checked & \checked\\
        6 & X & \checked & \checked\\
        7 & X & X & \checked\\
        8 & X & X & X\\
        9 & X & X & X\\
     
    \end{tabular}
  \end{center}
\caption{Aufgabe 2.4}
\end{table}


\end{document}

