\documentclass[a4paper, fontsize=10pt]{scrartcl}
\usepackage[utf8]{inputenc}
\usepackage[ngerman]{babel}
\usepackage{stmaryrd}
\usepackage{amsfonts}
\usepackage{amsmath}
\usepackage{mathpazo} %schickere Schriftart für Text
\usepackage{MnSymbol} %schickere Symbole
\usepackage{graphicx}
\usepackage{amsthm}
\usepackage{shadethm}
\usepackage[all,2cell,ps]{xy}
\usepackage{setspace}
\usepackage{listings}
\usepackage{color}		%Farben ermöglichen
\usepackage{colortbl} %Tabellen mit Farbe ermöglichen
\usepackage{hyperref} %Hyperlinks einfügbarisieren
\usepackage{fancyvrb} %verbatim mit mehr Variationen
\usepackage{listings} %Quellcode mit listings einbinden
\usepackage{moreverb}
\usepackage{tocloft} %tableofcontent mit Optionen
\usepackage{ulem} %neue Befehle für Unterstriche
\lstloadlanguages{Java} %Standardmäßig Java vorher laden
% Standard-Layout für die Code-Umgebung (alle Sprachen)
\lstset{%
      basicstyle=\footnotesize\ttfamily,
     showspaces=false,
     showtabs=false,
     columns=fixed,
     numbers=left,
     frame=none,
     numberstyle=\tiny,
     breaklines=true,
     showstringspaces=false,
     xleftmargin=1cm,
     tabsize=4
}%
\setlength\topmargin{-1cm}
\textheight 23cm
\textwidth 14cm
\setlength\oddsidemargin{1cm}
\setlength\evensidemargin{3.5cm}
\parindent=0pt
\setlength{\saveparindent}{\parindent}
%\pagestyle{empty}
\newcommand{\rem}[1]{} %rem-Kommentiermöglichkeit
\definecolor{dg}{rgb}{0.8,0.8,0.8}		%definiert dunkelgrau
\definecolor{hg}{rgb}{0.95,0.95,0.95}	%definiert hellgrau
%%%%%%%%%%%%%%%%%%%%%%%%%%%%%%%%%%%%%%%%%%%%%%%%%%%%%%%%%%%%%%%%%%%%%%%%%%%%%%%%%%%  
 
\begin{document} 

{\large Algorithmendesign \hfill Gruppe 2}\\  
{\large Lösungen zu Übungsblatt 2} \hfill Max Bannach\\
{\large WS 13/14}
\begin{flushright}Markus Richter (614027)\end{flushright}
\rule{\textwidth}{.3mm}

\section*{Aufgabe 2}
\subsection*{Teilaufgabe a)}
\begin{align*}
\cos+j\sin=irgendas
\\\Leftrightarrow j\sin+\cos=auch irgendwas
\end{align*}
\section*{Aufgabe 3}
$ x_1(n)=(-1)^n $\\
Divergent gegen $\pm1$, daher periodisch. Die kleinste Periode $T=2$, da $x_1(n)=x_1(n+2)$\bigskip

$ x_2(n)=j^n$\\
Imaginäre Zahl $j, j^2=-1$, daher kleinste Periode $T=2$\bigskip

$x_4(t)=2\sin(\pi t)+4\sin\left( \dfrac{2\pi t}{3}\right)$\\
Funktion besteht aus Summe von $\sin$-Funktionen, die $2\pi $-periodisch sind. Daher ist die Summe ebenfalls periodisch.\\
$\Rightarrow 2\sin (\pi t)$: $T_a=2$.\\$\Rightarrow 4\sin\left( \dfrac{2\pi t}{3}\right)$: $T_b=\dfrac{2\pi}{\dfrac {2\pi}{3}}=3$. Der größte gemeinsame Teiler ist $6$, daher $T=6$\bigskip

$x_4(t)=2\sin(\pi n)+4\sin\left( \dfrac{2\pi n}{3}\right)$. Analog zu $x_3$, der Definitionsbereich ändert nicht an der Periodizität.\bigskip

$x_5(t)=4\sin (t)+5\sin (3t)$\\
Wieder Summe von zwei $2\pi$-periodischen $\sin$-Funktionen. Daher ebenfalls periodisch.\\
$4\sin (t):$ gilt offensichtlich $T_a=2\pi$.\\$5\sin (3t): T_b=\dfrac{2\pi}{3} \Rightarrow T=2\pi$\bigskip

$x_6(n)=5\sin (n)+6\sin (3n)$.\\
Die kleinste Periode $T$ wäre hier auch $2\pi $ allerdings gilt $\pi \notin \mathbb{Z}\Rightarrow$ Keine Periodizität. 


\section*{Aufgabe 4}



\end{document}
